\chapter{Working with generated code}

\section{Compiling the IDL file}

\IDL can generate three types of output from a given interface description:

\begin{itemize}
\item \emph{Client stubs}, which are linked with every client application,
i.e. every application that needs to invoke methods from the interface
\item \emph{Server stubs}, which are used by every server that needs to
implement the interface.
\item \emph{Server templates}, which essentially contain a dummy implementation
for the interface. They can be used as a starting point for writing a server.
\end{itemize}

Usually, the first two kinds of output are generated during the compilation
process, whereas the third is only generated once and then extended with
user code. You can select the kind of output by supplying \texttt{-c},
\texttt{-s} or \texttt{-t} on the command line, respectively. For example,

\begin{center}\texttt{
idl4 -ix0 -pia32 -ffastcall -s pager.idl -h pager-server.h
}\end{center}

generates a header file called \texttt{pager-server.h} which contains all
the server stubs for methods in \texttt{pager.idl}. Also, the X0 backend
for the IA32 platform is selected, e.g. because the application will use
the Hazelnut kernel and run on a Pentium-III processor. Finally, the
\texttt{fastcall} option is given, which allows \IDL to use nonstandard
system calls, in this case, the \texttt{sysenter} instruction.

\section{Sending requests}

As specified by the CORBA C language mapping \cite{corba-clm}, client stubs
have two implicit parameters (i.e. parameters that are not defined by
the interface). Consider the following example:

\begin{center}\begin{tabular}{l@{\hspace{.4cm}}|@{\hspace{.5cm}}l}
\begin{minipage}{7cm}\small\begin{verbatim}
module storage {
   interface textfile {
      void readln(
        inout short pos, 
        out string line
      );
   };
};
\end{verbatim}\end{minipage} & 
\begin{minipage}{7cm}\small\begin{verbatim}
library storage {
   interface textfile {
      void readln(
        [in, out] short *pos, 
        [out, string] char **line
      );
   };
};
\end{verbatim}\end{minipage} \\
\end{tabular}\end{center}

\noindent When this definition is compiled with the \texttt{-c} option, \IDL creates
the following client stub:

\begin{center}\begin{minipage}{10cm}\small\begin{verbatim}
void storage_textfile_readln(
         storage_textfile _service,
         CORBA_short *pos, CORBA_char **line,
         CORBA_Environment *_env
)
\end{verbatim}\end{minipage}\end{center}

The first parameter, \texttt{\_service}, contains the thread ID of the server 
where the request is to be sent. Unlike other CORBA code generators, \IDL
does not provide a way to obtain this automatically; it assumes that this
functionality is implemented by your system.

The last parameter is a pointer to a \texttt{CORBA\_Environment} structure.
This structure contains additional information related to the call, such as
a timeout value or a memory window for receiving mappings. You must initialize
this structure before invoking the call. 

An invocation of \texttt{readln} could look like this:

\begin{center}\begin{minipage}{13cm}\small\begin{verbatim}
#include <storage_client.h>
void test(l4_threadid_t server) {
   CORBA_Environment env = idl4_default_environment;
   short pos = 100;
   char *buf;
  
   storage_textfile_readln(server, &pos, &buf, &env);
  
   switch (env._major) {
      case CORBA_SYSTEM_EXCEPTION: 
         printf("IPC failed, code %d\n", 
            CORBA_exception_id(&env));
         CORBA_exception_free(&env);
         return -1;
      case CORBA_USER_EXCEPTION:
         printf("User-defined exception");
         CORBA_exception_free(&env);
         return -1;
      case CORBA_NO_EXCEPTION:
         break;
   }

   printf("Read: %s\n", buf);
   CORBA_free(buf);
}
\end{verbatim}\end{minipage}\end{center}

Note how \texttt{env} is initialized with an \IDL-supplied default value,
which means a timeout of infinity and an empty receive window. Also, the
example shows how environment structure can be used to determine whether
an exception occurred during the call, and what type it was. Finally,
it is important to always release storage allocated by the stubs (e.g. 
exceptions, output strings, ...) with the appropriate function.

\section{Processing requests}

The standard server loop, which is included in the server template, mainly consists
of two macros:

\begin{itemize}
\item \texttt{idl4\_reply\_and\_wait}, which sends any pending replies and then
receives one incoming request, and
\item \texttt{idl4\_process\_request}, which analyzes the request and calls the
appropriate implementation function
\end{itemize}

Between those macros, you can insert additional code, e. g. a permission check.
The second macro uses a function table to decide which
function should handle the request; it takes the method number as an argument
and uses it as an index into the table. Table entries which correspond to unassigned
method numbers contain a reference to the \texttt{discard} function of the 
interface; thus, this function is only called when a malformed request was received.

The server template file also contains function templates for every method in
the interface. For the example interface from above, the following template
is created:

\begin{center}\begin{minipage}{13cm}\small\begin{verbatim}
IDL4_INLINE void storage_textfile_readln_implementation(
   CORBA_Object _caller, CORBA_short *pos, 
   CORBA_char **line, idl4_server_environment *_env
)

{
   /* implementation of storage::textfile::readln */
  
   return;
}

IDL4_PUBLISH_STORAGE_TEXTFILE_READLN(
   storage_textfile_readln_implementation
);
\end{verbatim}\end{minipage}\end{center}

Similar to the client side, the function has two additional parameters. The first
parameter, \texttt{\_caller}, contains the ID of the thread that has sent the
request, whereas the last parameter, \texttt{\_env}, points to an internal data
structure. Many functions in the \IDL runtime need a pointer to this structure,
for example the function \texttt{CORBA\_exception\_set}, which is used to 
raise exceptions.

Please note the macro at the end of the function. This macro makes the function
accessible to the server loop; for the optimizations to work, it is also very 
important that this macro is included \emph{after} the implementation function. 

Unlike the client side, memory allocation on the server side is mostly handled by
the stub code, that is, you do not need to call \texttt{CORBA\_free}, except if 
you explicitly allocated additional buffers. The stub code preallocates a buffer
for every \texttt{out} value and passes a pointer to the implementation 
function in the respective argument. However, for large buffers 
(such as strings), you can save one copy by overwriting the pointer with another
one pointing directly to the data you want to send.

If you decide not to send a reply at all, you can use the \texttt{idl4\_set\_no\_response}
function. In this case, the stub code will discard any \texttt{inout} or \texttt{out}
values and skip the send phase; instead, it will directly receive another request.
